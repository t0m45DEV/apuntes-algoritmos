\documentclass{article}
\usepackage{graphicx} % Imagenes
\usepackage{amssymb} % Operadores logicos
\usepackage[makeroom]{cancel} % Para tachar simbolitos
\usepackage{multicol} % Muchas columnas
\usepackage[margin=3cm]{geometry} % Margenes
\usepackage{amsmath}

\usepackage{listings} % Codigo
\usepackage{multicol} % Muchas columnas

\usepackage{tikz}
\usetikzlibrary{graphs,graphdrawing,arrows.meta, calc} % Arboles binarios
\usegdlibrary{trees}
\usetikzlibrary{shapes.geometric}
\usetikzlibrary{automata, arrows}

\newcommand{\In}{\textit{in }}
\newcommand{\Out}{\textit{out }}
\newcommand{\Inout}{\textit{inout }}

\newcommand{\tocarEspacios}{%
	\addtolength{\leftskip}{3em}%
	\setlength{\parindent}{0em}%
}

\newcommand{\comentario}[1]{$/*$ \textit{#1} $*/$}

\newenvironment{proc}[2]{

	\newcommand{\requiere}[1]{
		\texttt{requiere} $\{$ ##1 $\}$
	}

	\newcommand{\asegura}[1]{
		\texttt{asegura} $\{$ ##1 $\}$
	}

	\paragraph{} \vspace*{-5mm}
	\vspace{1ex}
	\texttt{proc} \textit{#1}(#2)$\{$
	\par
	\tocarEspacios
}
{

\hspace{2.5mm} $\}$
\vspace{2ex}
}


\newenvironment{tad}[1]{
	\paragraph{} \vspace*{-4mm}
	\newcommand{\obs}[2]{\texttt{obs} ##1 : ##2}

	\vspace{1ex}
	\texttt{TAD} \textit{#1} $\{$
	\par
	\tocarEspacios
}
{

\hspace{2.5mm} $\}$
\vspace{2ex}
}


\newenvironment{pred}[2]{
	\paragraph{}\vspace*{-4mm}
	\vspace{1ex}
	\texttt{pred} \textit{#1} ( #2 ) $\{$
	\par
	\tocarEspacios
}
{

\hspace{2.5mm} $\}$
\vspace{2ex}
}


\newcommand{\aux}[4]{\vspace{2pt} \texttt{aux} \textit{#1} (#2) : #3 $=$ #4 \vspace{2pt}}

\newcommand{\smallLang}[1]{
	\setlength\topsep{1pt}
	\setlength\parskip{1pt}
	\texttt{#1}
}

\newcommand{\renombre}[2]{\texttt{#1 es #2}}

\newenvironment{modulo}[2]{
	\paragraph{} \vspace*{-4mm}
	\newcommand{\var}[2]{##1 : ##2}

	\vspace{1ex}
	\texttt{Modulo} \textit{#1} \texttt{implementa} \textit{#2} $\{$
	\par
	\tocarEspacios
}
{

\hspace{2.5mm} $\}$
\vspace{2ex}
}

\newcommand{\tab}{\hspace*{4mm}}

\newcommand{\state}[1]{
	\setlength\topsep{1pt}
	\setlength\parskip{1pt}
	{$\{$ #1 $\}$}
}

\newcommand{\centrado}[1]{
	\setlength\topsep{1pt}
	\setlength\parskip{1pt}
	\begin{center}
		{#1}
	\end{center}
}


\renewcommand{\lstlistingname}{Código}
\lstset{% general command to set parameter(s)
	language=Java,
	morekeywords={endif, endwhile, skip},
	basewidth={0.47em,0.40em},
	columns=fixed, fontadjust, resetmargins, xrightmargin=5pt, xleftmargin=15pt,
	flexiblecolumns=false, tabsize=4, breaklines, breakatwhitespace=false, extendedchars=true,
	numbers=left, numberstyle=\tiny, stepnumber=1, numbersep=9pt,
	frame=l, framesep=3pt,
	captionpos=b,
}

\newcommand{\arbol}[1]{\begin{tikzpicture}[every node/.style={circle, draw=black}, subArbol/.style={regular polygon, regular polygon sides=3}, tips=false]
\graph[binary tree layout]{#1};
\end{tikzpicture}}

\newcommand{\arbolCentrado}[1]{\begin{center} \arbol{#1} \end{center}}


\newcommand{\variable}[3][]{#1$#2:$ #3}

\newcommand{\nat}{$\mathbb{N}$}
\newcommand{\ent}{$\mathbb{Z}$}
\newcommand{\float}{$\mathbb{R}$}
\newcommand{\bool}{$Bool$}
\newcommand{\cha}{$Char$}
\newcommand{\str}{$String$}

\newcommand{\indiceValido}[2][i]{$0 \leq #1 < |#2|$}

\newcommand{\paraTodoIndiceValido}[2][i]{$($\variable[$\forall$]{#1}{\ent}$)($\indiceValido[#1]{#2}}

\newcommand{\existeIndiceValido}[2][i]{$($\variable[$\exists$]{#1}{\ent}$)($\indiceValido[#1]{#2}}

\newcommand{\formula}[1]{$#1$}

\newcommand{\TLista}[1]{$seq \langle$#1$\rangle$}
\newcommand{\lista}[1]{$\langle$#1$\rangle$}
\newcommand{\head}[1]{\textit{head}(#1)}
\newcommand{\tail}[1]{\textit{tail}(#1)}
\newcommand{\addFirst}[2]{\textit{addFirst}(#1, #2)}
\newcommand{\subseq}[3]{\textit{subseq}(#1, #2, #3)}
\newcommand{\setAt}[3]{\textit{setAt}(#1, #2, #3)}

\newcommand{\set}[1]{$\{$#1$\}$}
\newcommand{\TConj}[1]{\formula{conj < #1 >}}

\newcommand{\TDict}[2]{\formula{dict < #1, #2 >}}

\newcommand{\TArray}[1]{\texttt{array[#1]}}

\newcommand{\True}{$True$}
\newcommand{\False}{$False$}
\newcommand{\Then}{$\rightarrow$}
\newcommand{\Iff}{$\longleftrightarrow$}
\newcommand{\implica}{\longrightarrow}
\newcommand{\IfThenElse}[3]{\texttt{if } #1 \texttt{ then } #2 \texttt{ else } #3 \texttt{ fi}}
\newcommand{\yLuego}{\land_{L}}
\newcommand{\oLuego}{$\lor_{L}$}
\newcommand{\implicaLuego}{$\longrightarrow_{L}$}

\newcommand{\enum}[2]{\texttt{enum} \textit{#1}$\{$#2$\}$}

\newcommand{\triplaDeHoare}[3]{\{#1\} \texttt{ #2 } \{#3\}}

\newcommand{\ejercicio}[2]{\paragraph{Ejercicio #1.} #2}
\newcommand{\respuesta}[1]{\begin{snugshade*}#1\end{snugshade*}}

\begin{document}
\thispagestyle{empty}

\section*{Estructuras y complejidades}

\begin{multicols}{2}

\begin{center}
	\begin{large}Lista Enlazada\end{large}
	\vspace*{2mm}
	
	\begin{tabular}{|c|c|}
		\hline
		\textbf{Operación} & \textbf{Complejidad} \\
		\hline
		listaVacía & $O(1)$ \\
		longitud & $O(1)$ \\
		vacía & $O(1)$ \\
		agregarAdelante & $O(cp(e))$ \\
		agregarAtras & $O(cp(e))$ \\
		fin & $O(1)$ \\
		comienzo & $O(1)$ \\
		primero & $O(1)$ \\
		ultimo & $O(1)$ \\
		\hline
		obtener & $O(n)$ \\
		eliminar & $O(n)$ \\
		modificarPosicion & $O(n + cp(e))$ \\
		concatenar & $O(m * cp(e))$ \\
		\hline
	\end{tabular}
\end{center}


\begin{center}
	\begin{large}Pila Sobre Lista\end{large}
	\vspace*{2mm}
	
	\begin{tabular}{|c|c|}
		\hline
		\textbf{Operación} & \textbf{Complejidad} \\
		\hline
		pilaVacía & $O(1)$ \\
		vacía & $O(1)$ \\
		encolar & $O(cp(e))$ \\
		desencolar & $O(1)$ \\
		tope & $O(1)$ \\
		\hline
	\end{tabular}
\end{center}


\begin{center}
	\begin{large}Cola Sobre Lista\end{large}
	\vspace*{2mm}
	
	\begin{tabular}{|c|c|}
		\hline
		\textbf{Operación} & \textbf{Complejidad} \\
		\hline
		colaVacía & $O(1)$ \\
		vacía & $O(1)$ \\
		encolar & $O(cp(e))$ \\
		desencolar & $O(1)$ \\
		proximo & $O(1)$ \\
		\hline
	\end{tabular}
\end{center}


\begin{center}
	\begin{large}Vector\end{large}
	\vspace*{2mm}
	
	\begin{tabular}{|c|c|}
		\hline
		\textbf{Operación} & \textbf{Complejidad} \\
		\hline
		vectorVacío & $O(1)$ \\
		longitud & $O(1)$ \\
		vacía & $O(1)$ \\
		agregarAdelante & $O(f(n) + cp(e))$ \\
		agregarAtras & $O(f(n) + cp(e))$ \\
		fin & $O(n)$ \\
		comienzo & $O(n)$ \\
		primero & $O(1)$ \\
		ultimo & $O(1)$ \\
		\hline
		obtener & $O(1)$ \\
		eliminar & $O(n)$ \\
		modificarPosicion & $O(1)$ \\
		concatenar & $O(m)$ \\
		\hline
	\end{tabular}
\end{center}


\begin{center}
	\begin{large}Conjunto Lineal\end{large}
	\vspace*{2mm}
	
	\begin{tabular}{|c|c|}
		\hline
		\textbf{Operación} & \textbf{Complejidad} \\
		\hline
		conjVacío & $O(1)$ \\
		tamaño & $O(1)$ \\
		pertenece & $O(n * eq(T))$ \\
		agregar & $O(n * eq(T) + cp(e))$ \\
		agregarRapido & $O(cp(e))$ \\
		sacar & $O(n * eq(T))$ \\
		unir & $O(n * m * cp(T) * eq(T))$ \\
		restar & $O(n * m * eq(T))$ \\
		intersecar & $O(n * m * eq(T))$ \\
		\hline
	\end{tabular}
\end{center}


\begin{center}
	\begin{large}Conjunto Log\end{large}
	\vspace*{2mm}
	
	\begin{tabular}{|c|c|}
		\hline
		\textbf{Operación} & \textbf{Complejidad} \\
		\hline
		conjVacío & $O(1)$ \\
		tamaño & $O(1)$ \\
		pertenece & $O(\log n * eq(T))$ \\
		agregar & $O(\log n * eq(T) + cp(e))$ \\
		agregarRapido & $O(\log n * eq(T) + cp(e))$ \\
		sacar & $O(\log n * eq(T))$ \\
		unir & $O((n + m)\log (n + m) * cp(T) * eq(T))$ \\
		restar & $O((n + m)\log (n + m) * eq(T))$ \\
		intersecar & $O((n + m)\log (n + m) * cp(T) * eq(T))$ \\
		\hline
	\end{tabular}
\end{center}


\begin{center}
	\begin{large}Conjunto Digital\end{large}
	\vspace*{2mm}
	
	\begin{tabular}{|c|c|}
		\hline
		\textbf{Operación} & \textbf{Complejidad} \\
		\hline
		conjVacío & $O(1)$ \\
		tamaño & $O(1)$ \\
		pertenece & $O(|e| * |a|)$ \\
		agregar & $O(|e| * |a| * eq(a) + cp(e))$ \\
		agregarRapido & $O(|e| * |a| * eq(a) + cp(e))$ \\
		sacar & $O(|e| * |a| * eq(a))$ \\
		unir & $O(m * max(c2) * |a| * eq(a))$ \\
		restar & $O(n * m * |a|)$ \\
		intersecar & $O(n * max(c1) * m * max(c2))$ \\
		\hline
	\end{tabular}
\end{center}


\begin{center}
	\begin{large}Diccionario Lineal\end{large}
	\vspace*{2mm}
	
	\begin{tabular}{|c|c|}
		\hline
		\textbf{Operación} & \textbf{Complejidad} \\
		\hline
		dictVacío & $O(1)$ \\
		está & $O(n)$ \\
		definir & $O(n * eq(K) + cp(k) + cp(v))$ \\
		definirRapido & $O(n * eq(K) + cp(k) + cp(v))$ \\
		obtener & $O(n)$ \\
		borrar & $O(n)$ \\
		tamaño & $O(1)$ \\
		\hline
	\end{tabular}
\end{center}


\begin{center}
	\begin{large}Diccionario Log\end{large}
	\vspace*{2mm}
	
	\begin{tabular}{|c|c|}
		\hline
		\textbf{Operación} & \textbf{Complejidad} \\
		\hline
		dictVacío & $O(1)$ \\
		está & $O(\log n * eq(K))$ \\
		definir & $O(\log n * eq(K) + cp(k) + cp(v))$ \\
		definirRapido & $O(\log n * eq(K) + cp(k) + cp(v))$ \\
		obtener & $O(\log n * eq(K))$ \\
		borrar & $O(\log n * eq(K))$ \\
		tamaño & $O(1)$ \\
		\hline
	\end{tabular}
\end{center}


\begin{center}
	\begin{large}Diccionario Digital\end{large}
	\vspace*{2mm}
	
	\begin{tabular}{|c|c|}
		\hline
		\textbf{Operación} & \textbf{Complejidad} \\
		\hline
		dictVacío & $O(1)$ \\
		está & $O(|k| * |a| * eq(a))$ \\
		definir & $O(|k| * |a| * eq(a) + cp(k) + cp(v))$ \\
		definirRapido & $O(|k| * |a| * eq(a) + cp(k) + cp(v))$ \\
		obtener & $O(|k| * |a| * eq(a))$ \\
		borrar & $O(|k| * |a| * eq(a))$ \\
		tamaño & $O(1)$ \\
		\hline
	\end{tabular}
\end{center}

\end{multicols}

\begin{center}
	\begin{large}Referencias\end{large}
	\vspace*{2mm}
	
	\begin{tabular}{|c|c|}
		\hline
		\textbf{Nombre} & \textbf{Explicacion} \\
		\hline
		$n$ & Cantidad de elementos de la estructura \\
		\hline
		$m$ & Cantidad de elementos de una segunda estructura \\
		\hline
		$cp(e)$ & Costo de copiar el elemento $e$ \\
		\hline
		$f(n)$ &  \formula{f(n) = 
    				\begin{cases}
    					n &\quad\text{si } n = 2^{k} \text{para algún k}\\
    					1 &\quad\text{en caso contrario} 
    				\end{cases}}\\
		\hline
		$eq(T)$ & Costo de comparar elementos de tipo $T$ \\
		\hline
		$cp(T)$ & Costo de copiar elementos de tipo $T$ \\
		\hline
		$max(c1)$ & La clave mas grande del primer conjunto \\
		\hline
		$max(c2)$ & La clave mas grande del segundo conjunto \\
		\hline
		$eq(K)$ & Costo de comparar claves definidas en el diccionario \\
		\hline
		$cp(k)$ & Costo de copiar la clave \\
		\hline
		$cp(v)$ & Costo de copiar el valor \\
		\hline
	\end{tabular}
\end{center}

\section*{Sorting}

\begin{center}
	\begin{large}Algoritmos de ordenamiento\end{large}
	\vspace*{2mm}

	\begin{tabular}{|c|c|c|c|c|}
		\hline
		\textbf{Nombre} & \textbf{Mejor caso} & \textbf{Caso promedio} & \textbf{Peor caso} & \textbf{Es estable} \\
		\hline
		\textit{Selection} & \formula{O(n^{2})} & \formula{O(n^{2})} & \formula{O(n^{2})} & No\\
		\hline
		\textit{Insertion} & \formula{O(n)} & \formula{O(n^{2})} & \formula{O(n^{2})} & Si\\
		\hline
		\textit{Merge} & \formula{O(n \log (n))} & \formula{O(n \log (n))} & \formula{O(n \log (n))} & Si\\
		\hline
		\textit{Quick} & \formula{O(n \log (n))} & \formula{O(n \log (n))} & \formula{O(n^{2})} & No\\
		\hline
		\textit{Bubble} & \formula{O(n)} & \formula{O(n^{2})} & \formula{O(n^{2})} & Si\\
		\hline
		\textit{Heap} & \formula{O(n \log (n))} & \formula{O(n \log (n))} & \formula{O(n \log (n))} & No\\
		\hline
	\end{tabular}
\end{center}

\section*{TADs}

\begin{center}\begin{large}Invariante de representación y Función de abstracción\end{large}\end{center}
	\vspace*{2mm}

\begin{pred}{invRep}{\variable[\In]{implementacion}{\texttt{ImplementacionDeTad}}}
	Todas las condiciones que las variables deban cumplir para que tu implementacion sea valida
\end{pred}

\begin{proc}{funAbs}{\variable[\In]{implementacion}{\texttt{ImplementacionDeTad}}, \variable[\Out]{instanciaDeTad}{\texttt{Tad}}}
	Describir todos los observadores del TAD en base a las variables de tu implementacion (no hace falta usarlas todas, pero si abarcar al TAD completo)
\end{proc}

\thispagestyle{empty}
\end{document}
