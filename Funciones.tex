\usepackage{listings} % Codigo
\usepackage{multicol} % Muchas columnas

\usepackage{tikz}
\usetikzlibrary{graphs,graphdrawing,arrows.meta, calc} % Arboles binarios
\usegdlibrary{trees}
\usetikzlibrary{shapes.geometric}
\usetikzlibrary{automata, arrows}

\newcommand{\In}{\textit{in }}
\newcommand{\Out}{\textit{out }}
\newcommand{\Inout}{\textit{inout }}

\newcommand{\tocarEspacios}{%
	\addtolength{\leftskip}{3em}%
	\setlength{\parindent}{0em}%
}

\newcommand{\comentario}[1]{$/*$ \textit{#1} $*/$}

\newenvironment{proc}[2]{

	\newcommand{\requiere}[1]{
		\texttt{requiere} $\{$ ##1 $\}$
	}

	\newcommand{\asegura}[1]{
		\texttt{asegura} $\{$ ##1 $\}$
	}

	\paragraph{} \vspace*{-5mm}
	\vspace{1ex}
	\texttt{proc} \textit{#1}(#2)$\{$
	\par
	\tocarEspacios
}
{

\hspace{2.5mm} $\}$
\vspace{2ex}
}


\newenvironment{tad}[1]{
	\paragraph{} \vspace*{-4mm}
	\newcommand{\obs}[2]{\texttt{obs} ##1 : ##2}

	\vspace{1ex}
	\texttt{TAD} \textit{#1} $\{$
	\par
	\tocarEspacios
}
{

\hspace{2.5mm} $\}$
\vspace{2ex}
}


\newenvironment{pred}[2]{
	\paragraph{}\vspace*{-4mm}
	\vspace{1ex}
	\texttt{pred} \textit{#1} ( #2 ) $\{$
	\par
	\tocarEspacios
}
{

\hspace{2.5mm} $\}$
\vspace{2ex}
}


\newcommand{\aux}[4]{\vspace{2pt} \texttt{aux} \textit{#1} (#2) : #3 $=$ #4 \vspace{2pt}}

\newcommand{\smallLang}[1]{
	\setlength\topsep{1pt}
	\setlength\parskip{1pt}
	\texttt{#1}
}

\newcommand{\renombre}[2]{\texttt{#1 es #2}}

\newenvironment{modulo}[2]{
	\paragraph{} \vspace*{-4mm}
	\newcommand{\var}[2]{##1 : ##2}

	\vspace{1ex}
	\texttt{Modulo} \textit{#1} \texttt{implementa} \textit{#2} $\{$
	\par
	\tocarEspacios
}
{

\hspace{2.5mm} $\}$
\vspace{2ex}
}

\newcommand{\tab}{\hspace*{4mm}}

\newcommand{\state}[1]{
	\setlength\topsep{1pt}
	\setlength\parskip{1pt}
	{$\{$ #1 $\}$}
}

\newcommand{\centrado}[1]{
	\setlength\topsep{1pt}
	\setlength\parskip{1pt}
	\begin{center}
		{#1}
	\end{center}
}


\renewcommand{\lstlistingname}{Código}
\lstset{% general command to set parameter(s)
	language=Java,
	morekeywords={endif, endwhile, skip},
	basewidth={0.47em,0.40em},
	columns=fixed, fontadjust, resetmargins, xrightmargin=5pt, xleftmargin=15pt,
	flexiblecolumns=false, tabsize=4, breaklines, breakatwhitespace=false, extendedchars=true,
	numbers=left, numberstyle=\tiny, stepnumber=1, numbersep=9pt,
	frame=l, framesep=3pt,
	captionpos=b,
}

\newcommand{\arbol}[1]{\begin{tikzpicture}[every node/.style={circle, draw=black}, subArbol/.style={regular polygon, regular polygon sides=3}, tips=false]
\graph[binary tree layout]{#1};
\end{tikzpicture}}

\newcommand{\arbolCentrado}[1]{\begin{center} \arbol{#1} \end{center}}


\newcommand{\variable}[3][]{#1$#2:$ #3}

\newcommand{\nat}{$\mathbb{N}$}
\newcommand{\ent}{$\mathbb{Z}$}
\newcommand{\float}{$\mathbb{R}$}
\newcommand{\bool}{$Bool$}
\newcommand{\cha}{$Char$}
\newcommand{\str}{$String$}

\newcommand{\indiceValido}[2][i]{$0 \leq #1 < |#2|$}

\newcommand{\paraTodoIndiceValido}[2][i]{$($\variable[$\forall$]{#1}{\ent}$)($\indiceValido[#1]{#2}}

\newcommand{\existeIndiceValido}[2][i]{$($\variable[$\exists$]{#1}{\ent}$)($\indiceValido[#1]{#2}}

\newcommand{\formula}[1]{$#1$}

\newcommand{\TLista}[1]{$seq \langle$#1$\rangle$}
\newcommand{\lista}[1]{$\langle$#1$\rangle$}
\newcommand{\head}[1]{\textit{head}(#1)}
\newcommand{\tail}[1]{\textit{tail}(#1)}
\newcommand{\addFirst}[2]{\textit{addFirst}(#1, #2)}
\newcommand{\subseq}[3]{\textit{subseq}(#1, #2, #3)}
\newcommand{\setAt}[3]{\textit{setAt}(#1, #2, #3)}

\newcommand{\set}[1]{$\{$#1$\}$}
\newcommand{\TConj}[1]{\formula{conj < #1 >}}

\newcommand{\TDict}[2]{\formula{dict < #1, #2 >}}

\newcommand{\TArray}[1]{\texttt{array[#1]}}

\newcommand{\True}{$True$}
\newcommand{\False}{$False$}
\newcommand{\Then}{$\rightarrow$}
\newcommand{\Iff}{$\longleftrightarrow$}
\newcommand{\implica}{\longrightarrow}
\newcommand{\IfThenElse}[3]{\texttt{if } #1 \texttt{ then } #2 \texttt{ else } #3 \texttt{ fi}}
\newcommand{\yLuego}{\land_{L}}
\newcommand{\oLuego}{$\lor_{L}$}
\newcommand{\implicaLuego}{$\longrightarrow_{L}$}

\newcommand{\enum}[2]{\texttt{enum} \textit{#1}$\{$#2$\}$}

\newcommand{\triplaDeHoare}[3]{\{#1\} \texttt{ #2 } \{#3\}}

\newcommand{\ejercicio}[2]{\paragraph{Ejercicio #1.} #2}
\newcommand{\respuesta}[1]{\begin{snugshade*}#1\end{snugshade*}}